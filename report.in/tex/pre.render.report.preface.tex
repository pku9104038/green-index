\myprefacetitle{前~言}
\myprefacetext

教育质量是衡量一个国家、地区基础教育发展水平最重要的指标,学生学业质量是教育质量的重要组成部分。以学生学业质量的评价为切入口,全面关注学生的健康成长,引导建立正确的质量观,建立教育质量的保障体系,对于促进基础教育的均衡发展,实现教育公平有着重大作用。

教育质量的形成涉及诸多关键环节,除外部的支持保障条件外,提升质量首先应关注教学过程。但是,长期存在的以片面追求升学率为导向的教学、考试,偏离了课程改革的方向和课程标准的基本要求,考试命题过多依赖个人经验,缺乏对教育测量学的深入研究;考试结果用于排名、甄别和选拔,加剧了竞争。在此压力下,中小学生很难有发自内心的求知欲,很难有对学校的归属感。过重的学习压力,不仅使学生的睡眠时间得不到保证,近视率不断增长,而且导致学生产生焦虑,严重地影响体质健康。正如欧盟《学生学业成绩分析报告》中所指出的“尽管较强的竞争性可能会在学习成绩上带来收益,但在学生的动机和心理健康方面却会付出代价。这些代价从长远来看会有一些不可预见的负面影响,如对学生的终身学习”。

2010年,上海市教育委员会承担了国家教育体制改革试点项目《改革义务教育教学质量综合评价办法》。2011年,上海市教育委员会在与教育部基础教育课程教材发展中心进行前期合作的基础上,共同研究、提炼出了一系列影响学生学业质量的关键因素,如学习动力、师生关系、学习负担等,构建了以关注学生健康成长为核心价值追求的上海市中小学学业质量绿色指标(以下简称“绿色指标”)。


“绿色指标”分为学生学业水平指数、学生学习动力指数、学生学业负担指数、师生关系指数、教师教学方式指数、校长课程领导力指数、学生社会经济背景与学业成绩相关指数、学生品德行为指数、学生体质健康指数、跨年度进步指数等。


2011年、2012年,依据“绿色指标”,上海组织了两轮学业质量综合评价,产生了积极广泛的影响,成为上海基础教育转型发展的一个亮点,并被教育部向全国推广。


2014年,上海的“绿色指标”综合评价工作进入了全面自主化的新阶段,建立了“指标确立—工具研发—数据分析—报告撰写—反馈改进”的评价操作路径;形成了“检测依靠技术、结论源自证据、分析产生转变”教学实证研究模式。


“绿色指标”综合评价改革是一个不断完善的行动改进实践过程,近年来,上海各区、校通过评价指标解读、综合评价实施、评价报告发布、基于评价结果进行教育教学改进等工作,同时,也进一步探索和完善了“绿色指标”综合评价。2017年7月,在吸收和总结多年来“绿色指标”评价实践经验和理论研究成果的基础上,上海市教委教研室正在探索“绿色指标”1.0到2.0的修订,旨在通过更加合事实、合逻辑、合规范、合目的评价将基于评价结果的改进落实到学校、学生、教师行为改进上,推进“为了评价的教学”走向“为了教学的评价”。


学业质量的“绿色指标”,并不是全面衡量教育质量的完整指标体系,而是针对当前时弊提出的,直接指向学生健康成长的关键因素,该指标在使用过程中将不断地得到发展和完善。


“绿色指标”体系的推行将为各级教育行政部门、教研部门和学校了解学生学业质量的基本状况和重要影响因素,提供实证的依据;为教育决策提供重要参考;为提升学生学业质量提供诊断和改进建议;从而引导全社会树立正确的质量观,促进学生的健康成长!

\clearpage
