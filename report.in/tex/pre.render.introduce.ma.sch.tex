## 四年级数学学业水平描述


2016年度上海市中小学学业质量绿色指标数学学科测试数学学科的三个主要命题维度是:内容领域、能力维度和水平层次,具体如下:


### 内容领域


内容领域分为以下三个方面:“数与代数”、“图形与几何”、“统计与概率”。“数与代数”考查了“数的认识”、“数的运算”、“常见的量”、“探索规律”四个方面。“图形与几何”考查了“图形的认识”、“图形的运动”、“图形与位置”、“测量”四个方面。“统计与概率”考查了“数据统计活动初步”。


### 能力维度


能力维度则依据《标准》的目标体系、国际同类测试的惯例和国内课程实施现状,分为以下四个维度:事实性知识和简单技能(简称知识技能)、数学理解、运用规则、问题解决。


“知识技能”主要考查了解基本数学事实及使用基本技能,具体体现在记忆事实性结论和约定等,从具体情境中辨认出数学对象,根据法则进行计算和使用工具进行测量,用直尺等进行作图。


“数学理解”主要考查对数学对象及其联系的理解,具体体现在利用模型、实例、自然语言、图表、数或字母等表示数学对象;结合具体的情境对数学对象进行理解和解释,利用数学对象对具体情境中的现象进行解释;能根据数学对象的特征判断对象的属性,以及与其相关对象之间的区别和联系;根据标准将物体、图形、数和数据等进行分类,能正确地将某一对象进行归类等。


“运用规则”主要考查利用已掌握的数学对象解决常规问题,具体体现在选择应用概念、规则和方法等解决常规问题;对结果的意义进行解释与验证。


“问题解决”主要考查分析、选择或创造方法解决问题(主要指非常规问题),具体体现在根据具体情境中的信息发现和提出简单的数学问题,读懂问题情境中的数学信息,寻求与选择问题情境中的数学关系,并用适当的方式进行表达;在观察、操作等活动中,能提出一些简单的猜想,能进行简单的数学推理;尝试回顾学习过程和问题解决的过程,建立知识间的联系。
