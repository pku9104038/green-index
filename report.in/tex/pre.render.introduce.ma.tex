# 学科测试说明

## 四年级数学学科测试简介

2016年度上海市中小学学业质量绿色指标数学学科测试数学学科的三个主要命题维度是:内容领域、能力维度和水平层次,具体如下:

### 内容领域

内容领域分为以下三个方面:“数与代数”、“图形与几何”、“统计与概率”。“数与代数”考查了“数的认识”、“数的运算”、“常见的量”、“探索规律”四个方面。“图形与几何”考查了“图形的认识”、“图形的运动”、“图形与位置”、“测量”四个方面。“统计与概率”考查了“数据统计活动初步”。

### 能力维度

能力维度则依据《标准》的目标体系、国际同类测试的惯例和国内课程实施现状,分为以下四个维度:事实性知识和简单技能(简称知识技能)、数学理解、规则运用、问题解决。

“知识技能”主要考查了解基本数学事实及使用基本技能,具体体现在记忆事实性结论和约定等,从具体情境中辨认出数学对象,根据法则进行计算和使用工具进行测量,用直尺等进行作图。

“数学理解”主要考查对数学对象及其联系的理解,具体体现在利用模型、实例、自然语言、图表、数或字母等表示数学对象;结合具体的情境对数学对象进行理解和解释,利用数学对象对具体情境中的现象进行解释;能根据数学对象的特征判断对象的属性,以及与其相关对象之间的区别和联系;根据标准将物体、图形、数和数据等进行分类,能正确地将某一对象进行归类等。

“规则运用”主要考查利用已掌握的数学对象解决常规问题,具体体现在选择应用概念、规则和方法等解决常规问题;对结果的意义进行解释与验证。

“问题解决”主要考查分析、选择或创造方法解决问题(主要指非常规问题),具体体现在根据具体情境中的信息发现和提出简单的数学问题,读懂问题情境中的数学信息,寻求与选择问题情境中的数学关系,并用适当的方式进行表达;在观察、操作等活动中,能提出一些简单的猜想,能进行简单的数学推理;尝试回顾学习过程和问题解决的过程,建立知识间的联系。

## 四年级数学学业水平描述

`r tab_name <- "四年级学生数学学科学业水平描述" `
四年级数学测试主要从“知识技能”、“数学理解”、“规则运用”和“问题解决”四个能力维度考查学生的数学学业水平,表\ref{tab: `r tab_name`}对学生在数学学科不同考查能力上的表现分水平进行了描述。


\begin{itshape}
\small
\begin{longtable}{|c|c|p{12cm}|}
\caption{`r tab_name`} \label{tab: `r tab_name`} \\
 
\hline
 \multicolumn{1}{|c}{\normalsize 考查能力} &  \multicolumn{1}{|c|}{\normalsize 水平} &  \multicolumn{1}{c|}{\normalsize 水平描述} \\ 
\hline
   
   \multirow{9}{*}{知识技能}  & \multirow{3}{*}{A}  & ★ 能在联系或动态的情境中回忆事实性结论和约定及辨识数学对象;  \\    
   & &  ★ 能正确选择法则,进行操作或计算; \\  
   & &  ★ 能使用较为复杂的工具进行测量、根据给定的条件进行较为复杂的作图。 \\    
\cline{2-3}      

   &   \multirow{3}{*}{B}  &  ★ 能在情境中回忆事实性结论和约定及辨识数学对象;\\  
   & &  ★ 能根据法则,进行正确操作或计算;\\  
   & &  ★ 能使用简单工具进行直接测量、根据给定的条件进行简单作图。 \\    
\cline{2-3}      

   &   \multirow{3}{*}{C}  &  ★ 不能在情境中回忆事实性结论和约定及辨识数学对象,或回忆出部分、混淆简单事实性结论、对象;\\  
   & &  ★ 不能根据法则进行操作或计算,或操作与计算中频繁出错;\\  
   & &  ★ 不能用简单工具进行最基本的测量、作图,或者测量与作图中频繁出错。 \\   
\hline
   
   \multirow{3}{*}{数学理解}  & \multirow{1}{*}{A}  & ★  能够在模型、自然语言、图表、数或字母之间等进行转化; \\    
   & &  ★ 能用自己的语言准确描述数学对象的特征,利用数学对象的特征对复杂情境中的现象进行解释; \\  
   & &  ★ 能识别出复杂情境中的数学对象,根据数学对象的意义、性质判断对象的属性以及与其相关对象之间的联系和区别;\\

   \hline    
   \multirow{9}{*}{数学理解}  &  &  ★ 能根据问题需要用两种或两种以上的标准对数学对象进行分类。\\    
\cline{2-3}      

   &   \multirow{4}{*}{B}  &  ★ 能用模型、实例、自然语言、图表、数或字母等多种方式表示数学对象;\\  
   & &  ★   能用自己的语言描述数学对象的特点,用数学对象的特征对简单情境中的现象进行解释;\\  
   & &  ★ 能识别出简单情境中的数学对象,并判断对象的属性; \\    
   & &  ★ 能根据问题需要自己确定一个标准对数学对象进行分类。 \\    
\cline{2-3}      

   &   \multirow{4}{*}{C}  &  ★ 不能选择适当的形式表示数学对象,或选用其中的一种方式表达不完整,或不能在不同形式之间进行简单转化;\\  
   & &  ★ 不能描述数学对象或用数学对象对简单情境中的现象进行解释,或描述、解释不完整,有明显错误;\\  
   & &  ★ 不能识别出简单情境中的数学对象及不能正确判断对象的属性,或识别的数学对象存在偏差,判断的属性有明显的错误; \\   
   & &  ★ 不能根据给定的标准对数学对象进行分类,或分类过程中出现混乱。 \\   
\hline
   
   \multirow{6}{*}{规则运用}  & \multirow{2}{*}{A}  & ★   在复杂情境中识别解决具体问题所需要的算法、法则、公式等,并通过列式计算、画出图表等解决常规问题; \\    
   & &  ★ 能对结果的意义进行解释,能根据意义验证结果的合理性。 \\   
\cline{2-3}      

   &   \multirow{2}{*}{B}  &  ★ 在简单情境中识别解决具体问题所需要的算法、法则、公式等,并通过列式计算、画出图表等解决常规问题;\\  
   & &  ★   能对结果的意义进行解释。\\   
\cline{2-3}      

   &   \multirow{2}{*}{C}  &  ★ 不能在简单情境中识别解决具体问题所需要的算法、法则、公式等,或者在运用法则、公式等时经常发生错误;\\  
   & &  ★ 不能对结果的意义进行解释,或解释明显缺乏合理性。\\  
\hline
   
   \multirow{13}{*}{问题解决}  & \multirow{5}{*}{A}  & ★ 能读懂复杂情境中的数学信息,对给定的信息进行合理的假设与推断; \\    
   & &  ★ 能利用生活现象、直观模型等进行简单推理,验证获得的结果和数学结论; \\  
   & &  ★ 能运用知识、方法等解决非常规问题;\\    
   & &  ★ 能对解决问题中的知识和方法进行讨论和评价,并进行简单的推广;\\  
   & &  ★ 能对过程与方法进行反思,初步建立知识之间的联系。\\    
\cline{2-3}      

   &   \multirow{4}{*}{B}  &  ★ 能读懂问题情境中的数学信息,从给定的信息中作出简单的假设与推断;\\  
   & &  ★   能通过举例等验证结果和数学结论;\\  
   & &  ★ 能运用知识、方法等解决简单的非常规问题; \\   
   
   & &  ★ 能根据要求对解决一类问题的知识与方法进行简单总结。 \\    
\hline
\hline      

   &   \multirow{4}{*}{C}  &  ★ 不能读懂问题情境中的数学信息,或不能根据问题有效提取问题情境中的数学信息;\\  
   & &  ★ 不能对数学结果和结论阐述自己的理由,或理由不合理;\\  
   & &  ★ 不能运用知识、方法解决简单的非常规问题,或解决问题的基本策略与方法有明显错误; \\   
   & &  ★ 不能根据要求对过程与方法进行简单总结,或总结中有明显错误。 \\   
    
\hline 
\end{longtable}
\end{itshape}
