## 四年级语文学业水平描述

2016年度上海市中小学学业质量绿色指标语文学科测试依据学科测试框架在内容领域和考查维度两方面来考查学生语文方面的学业水平。


语文学科测试在内容领域上分为积累、阅读和习作三个维度。其中在积累这一内容领域里,主要从学生读准字音、认清字形、理解词义、积累和运用常见古诗文等四个方面的能力维度,测评和分析学生的学业表现。阅读分为阅读测试情境和阅读测试能力,前者考察学生在获取信息、获取文学体验、完成任务三个方面的阅读能力;后者考察学生在提取信息、形成解释、整体感知和解决问题方面的阅读能力。习作主要从选择材料、组织材料、语言表达和书写及标点等四个方面对学生的习作水平进行具体分析。


语文学科测试的考查维度则依据《上海市中小学语文课程标准》、国际同类测试的惯例和国内课程实施现状,从知识和能力、过程和方法、情感态度和价值观三个维度全面考查学生的语文素养。对这三个维度的考查,可以反映出学生对语文基础知识和基本技能的掌握情况以及语言文字的综合运用能力,同时关注学生在语文学习过程中的感悟、体验和审美活动,体现语文学科工具性和人文性统一的基本特性。

`r tab_name <- "四年级语文学科测试框架"`

如表\ref{tab: `r tab_name`}所示,本次四年级语文测试主要考查学生“积累”“阅读”和“习作”三方面的能力。


\begin{table}[H]
\centering
\caption{`r tab_name`} \label{tab: `r tab_name`} 
\itshape
\mytable
\begin{tabularx}{\textwidth}{|c|l|X|}
\hline

  & \makecell[c]{\mytablehead{内容}} & \makecell[c]{\mytablehead{能力}} \\
\hline

 \multirow{9}{3cm}{\makecell[cc]{语文}} & \multirow{3}{3cm}{积累} & 掌握已学过的常用汉字 \\
 & & 具有独立识字的能力 \\
 & & 积累古诗 \\
\cline{2-3}

 & \multirow{2}{3cm}{阅读} & 初步具有现代文阅读能力 \\
 & & 具备基本的语文积累和综合运用能力 \\
\cline{2-3}

 & \multirow{3}{3cm}{\makecell[cl]{习作}} & 能根据写作要求选材立意 \\
 & & 能根据表达的需要合理安排文章结构 \\
 & & 能文从字顺地表达自己的意思 \\
 & & 能正确、规范、端正地书写汉字 \\

\hline

\end{tabularx}
\end{table}
