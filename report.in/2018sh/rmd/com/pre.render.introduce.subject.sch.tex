## 四年级语文学业水平描述

四年级语文测试主要从“积累”、“阅读”和“习作”三个方面考查学生的语文学业水平。
`r tab_name <- "四年级学生语文学科学业水平描述" `
表\ref{tab: `r tab_name`}对四年级学生在语文学科不同考查能力上的表现分水平进行了学业水平描述。


\begin{itshape}
\mytable
\begin{longtable}{|c|c|p{12cm}|}
\caption{`r tab_name`} \label{tab: `r tab_name`} \\

\hline
 \multicolumn{1}{|c}{\mytablehead{考查能力}} &  \multicolumn{1}{|c|}{\mytablehead{水平}} &  \multicolumn{1}{c|}{\mytablehead{水平描述}} \\ 
\hline
   \multirow{10}{*}{积累}  & \multirow{2}{*}{A}  & ★ 能够理解常用词语、成语的基本意思,并分辨词义间的细微差别  \\  
   & &  ★ 能够借助语感在较复杂的语境中恰当运用词语、成语 \\    
\cline{2-3}      

   &   \multirow{4}{*}{B}  &  ★	能够读准常用字中同年级学生易错字字音 \\  
   & &  ★	 能够识记常用字中同年级学生易错字字形 \\  
   & &  ★	 能正确理解常用词语的基本意思,并在语境中恰当运用 \\    
   & &  ★	 能正确记忆和理解学过的教材中的古诗 \\    
\cline{2-3}

   &   \multirow{4}{*}{C}  & ★ 不能正确分辨易错字字音;或只能利用汉语拼音读出所学过的部分常用字字音 \\  
   & &  ★	不能分辨易错字字形;或只能认清所学过的部分常用字字形 \\  
   & &  ★	不能在语境中恰当运用词语、成语;或只能理解所学过的部分常用词语、成语的基本意思 \\    
   & &  ★	不能正确理解古诗文的内容,不能根据上下句的提示准确记忆学过的教材中的古诗 \\ 

\hline
   \multirow{6}{*}{阅读}  &   \multirow{1}{*}{A}  & ★	能够综合利用文本的信息并联系个人经验从多个角度对问题做出合理的解释 \\    
\cline{2-3}      

   &   \multirow{3}{*}{B}  &  ★	能够根据要求从文本中提取出相关信息并进行简单的比较或概括 \\  
   & &  ★	能够通过对相关信息的简单加工对问题做出解释 \\  
   & &  ★	能够完整感知文本的主要内容 \\    
\cline{2-3}

   &   \multirow{2}{*}{C}  & ★	不能对所提取的信息做出加工;或只能从一个文段中提取直接陈述的信息,利用直接提取的信息进行解释 \\  
   & &  ★	不能完整感知文本的主要内容 \\
   
\hline
   \multirow{9}{*}{习作}  & \multirow{3}{*}{A}  & ★	能够根据给定的情境叙述一件事情或介绍一种事物,想象有新意,内容丰满  \\  
   & &  ★	能够合理、巧妙安排故事内容 \\    
   & &  ★	语句通顺,描写生动,标点正确 \\    
\cline{2-3}      

   &   \multirow{3}{*}{B}  &  ★	能够根据给定的情境叙述一件事情或介绍一种事物,结构基本完整 \\  
   & & ★	能较清楚地表达自己的意思,语言较为通顺 \\  
   & &  ★	有个别错别字,标点符号使用基本正确 \\    
\cline{2-3}

   &   \multirow{3}{*}{C}  & ★	不能够根据给定的情境作文;或能够根据给定的情境作文,但结构不完整;或内容过于简单 \\  
   & &  ★	不能清楚地表达自己的意思 \\  
   & &  ★	错别字较多,标点符号使用不规范 \\ 
   
\hline 
\end{longtable}
\end{itshape}


## 四年级数学学业水平描述

`r tab_name <- "四年级学生数学学科学业水平描述" `
四年级数学测试主要从“知识技能”、“数学理解”、“规则运用”和“问题解决”四个能力维度考查学生的数学学业水平,表\ref{tab: `r tab_name`}对学生在数学学科不同考查能力上的表现分水平进行了描述。


\begin{itshape}
\mytable
\begin{longtable}{|c|c|p{12cm}|}
\caption{`r tab_name`} \label{tab: `r tab_name`} \\
 
\hline
 \multicolumn{1}{|c}{\normalsize 考查能力} &  \multicolumn{1}{|c|}{\normalsize 水平} &  \multicolumn{1}{c|}{\normalsize 水平描述} \\ 
\hline
   
   \multirow{9}{*}{知识技能}  & \multirow{3}{*}{A}  & ★ 能在联系或动态的情境中回忆事实性结论和约定及辨识数学对象;  \\    
   & &  ★ 能正确选择法则,进行操作或计算; \\  
   & &  ★ 能使用较为复杂的工具进行测量、根据给定的条件进行较为复杂的作图。 \\    
\cline{2-3}      

   &   \multirow{3}{*}{B}  &  ★ 能在情境中回忆事实性结论和约定及辨识数学对象;\\  
   & &  ★ 能根据法则,进行正确操作或计算;\\  
   & &  ★ 能使用简单工具进行直接测量、根据给定的条件进行简单作图。 \\    
\cline{2-3}      

   &   \multirow{3}{*}{C}  &  ★ 不能在情境中回忆事实性结论和约定及辨识数学对象,或回忆出部分、混淆简单事实性结论、对象;\\  
   & &  ★ 不能根据法则进行操作或计算,或操作与计算中频繁出错;\\  
   & &  ★ 不能用简单工具进行最基本的测量、作图,或者测量与作图中频繁出错。 \\   
\hline
   
   \multirow{3}{*}{数学理解}  & \multirow{3}{*}{A}  & ★   能够在模型、自然语言、图表、数或字母之间等进行转化; \\    
   & &  ★ 能用自己的语言准确描述数学对象的特征,利用数学对象的特征对复杂情境中的现象进行解释; \\  
   & &  ★ 能识别出复杂情境中的数学对象,根据数学对象的意义、性质判断对象的属性以及与其相关对象之间的联系和区别;\\    
   \multirow{9}{*}{数学理解} & &  ★ 能根据问题需要用两种或两种以上的标准对数学对象进行分类。\\    
\cline{2-3}      

   &   \multirow{4}{*}{B}  &  ★ 能用模型、实例、自然语言、图表、数或字母等多种方式表示数学对象;\\  
   & &  ★   能用自己的语言描述数学对象的特点,用数学对象的特征对简单情境中的现象进行解释;\\  
   & &  ★ 能识别出简单情境中的数学对象,并判断对象的属性; \\ 
   
   & &  ★ 能根据问题需要自己确定一个标准对数学对象进行分类。 \\    
\cline{2-3}      

   &   \multirow{4}{*}{C}  &  ★ 不能选择适当的形式表示数学对象,或选用其中的一种方式表达不完整,或不能在不同形式之间进行简单转化;\\  
   & &  ★ 不能描述数学对象或用数学对象对简单情境中的现象进行解释,或描述、解释不完整,有明显错误;\\  
   & &  ★ 不能识别出简单情境中的数学对象及不能正确判断对象的属性,或识别的数学对象存在偏差,判断的属性有明显的错误; \\   
   & &  ★ 不能根据给定的标准对数学对象进行分类,或分类过程中出现混乱。 \\   
\hline
   
   \multirow{6}{*}{规则运用}  & \multirow{2}{*}{A}  & ★   在复杂情境中识别解决具体问题所需要的算法、法则、公式等,并通过列式计算、画出图表等解决常规问题; \\    
   & &  ★ 能对结果的意义进行解释,能根据意义验证结果的合理性。 \\   
\cline{2-3}      

   &   \multirow{2}{*}{B}  &  ★ 在简单情境中识别解决具体问题所需要的算法、法则、公式等,并通过列式计算、画出图表等解决常规问题;\\  
   & &  ★   能对结果的意义进行解释。\\   
\cline{2-3}      

   &   \multirow{2}{*}{C}  &  ★ 不能在简单情境中识别解决具体问题所需要的算法、法则、公式等,或者在运用法则、公式等时经常发生错误;\\  
   & &  ★ 不能对结果的意义进行解释,或解释明显缺乏合理性。\\  
\hline
   
   \multirow{13}{*}{问题解决}  & \multirow{5}{*}{A}  & ★ 能读懂复杂情境中的数学信息,对给定的信息进行合理的假设与推断; \\    
   & &  ★ 能利用生活现象、直观模型等进行简单推理,验证获得的结果和数学结论; \\  
   & &  ★ 能运用知识、方法等解决非常规问题;\\    
   & &  ★ 能对解决问题中的知识和方法进行讨论和评价,并进行简单的推广;\\  
   & &  ★ 能对过程与方法进行反思,初步建立知识之间的联系。\\    
\cline{2-3}      

    &   \multirow{4}{*}{B}  &  ★ 能读懂问题情境中的数学信息,从给定的信息中作出简单的假设与推断;\\  
   & &  ★   能通过举例等验证结果和数学结论;\\  
   & &  ★ 能运用知识、方法等解决简单的非常规问题; \\   
   
   & &  ★ 能根据要求对解决一类问题的知识与方法进行简单总结。 \\        
\cline{2-3}   
   &   \multirow{4}{*}{C}  &  ★ 不能读懂问题情境中的数学信息,或不能根据问题有效提取问题情境中的数学信息;\\  
   & &  ★ 不能对数学结果和结论阐述自己的理由,或理由不合理;\\  
   & &  ★ 不能运用知识、方法解决简单的非常规问题,或解决问题的基本策略与方法有明显错误; \\   
   & &  ★ 不能根据要求对过程与方法进行简单总结,或总结中有明显错误。 \\   
    
\hline 
\end{longtable}
\end{itshape}

