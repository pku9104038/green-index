# 学科测试说明

## 四年级语文学科测试简介

2016年度上海市中小学学业质量绿色指标语文学科测试依据学科测试框架在内容领域和考查维度两方面来考查学生语文方面的学业水平。

语文学科测试在内容领域上分为积累、阅读和习作三个维度。其中在积累这一内容领域里,主要从学生读准字音、认清字形、理解词义、积累和运用常见古诗文等四个方面的能力维度,测评和分析学生的学业表现。阅读主要考察学生提取信息、形成解释、整体感知和解决问题等四个方面。习作主要从选择材料、组织材料、语言表达和书写及标点等四个方面对学生的习作水平进行具体分析。


语文学科测试的考查维度则依据《上海市中小学语文课程标准》、国际同类测试的惯例和国内课程实施现状,从知识和能力、过程和方法、情感态度和价值观三个维度全面考查学生的语文素养。对这三个维度的考查,可以反映出学生对语文基础知识和基本技能的掌握情,以及语言文字的综合运用能力,同时关注学生在语文学习过程中的感悟、体验和审美活动,体现语文学科工具性和人文性统一的基本特性。

## 四年级语文学业水平描述

四年级语文测试主要从“积累”、“阅读”和“习作”三个方面考查学生的语文学业水平。
`r tab_name <- "四年级学生语文学科学业水平描述" `
表\ref{tab: `r tab_name`}对四年级学生在语文学科不同考查能力上的表现分水平进行了学业水平描述。

\begin{table}[H]
\centering
\caption{`r tab_name`} \label{tab: `r tab_name`} 

\itshape
\mytable
\begin{tabularx}{\textwidth}{|c|c|X|}

\hline
 \multicolumn{1}{|c}{\mytablehead{考查能力}} &  \multicolumn{1}{|c|}{\mytablehead{水平}} &  \multicolumn{1}{c|}{\mytablehead{水平描述}} \\ 
\hline
   \multirow{10}{*}{积累}  & \multirow{2}{*}{A}  & ★ 能够理解常用词语、成语的基本意思,并分辨词义间的细微差别  \\  
   & &  ★ 能够借助语感在较复杂的语境中恰当运用词语、成语 \\    
\cline{2-3}      

   &   \multirow{4}{*}{B}  &  ★	能够读准常用字中同年级学生易错字字音 \\  
   & &  ★	 能够识记常用字中同年级学生易错字字形 \\  
   & &  ★	 能正确理解常用词语的基本意思,并在语境中恰当运用 \\    
   & &  ★	 能正确记忆和理解学过的教材中的古诗 \\    
\cline{2-3}

   &   \multirow{4}{*}{C}  & ★ 不能正确分辨易错字字音;或只能利用汉语拼音读出所学过的部分常用字字音 \\  
   & &  ★	不能分辨易错字字形;或只能认清所学过的部分常用字字形 \\  
   & &  ★	不能在语境中恰当运用词语、成语;或只能理解所学过的部分常用词语、成语的基本意思 \\    
   & &  ★	不能正确理解古诗文的内容,不能根据上下句的提示准确记忆学过的教材中的古诗 \\ 

\hline
   \multirow{6}{*}{阅读}  &   \multirow{1}{*}{A}  & ★	能够综合利用文本的信息并联系个人经验从多个角度对问题做出合理的解释 \\    
\cline{2-3}      

   &   \multirow{3}{*}{B}  &  ★	能够根据要求从文本中提取出相关信息并进行简单的比较或概括 \\  
   & &  ★	能够通过对相关信息的简单加工对问题做出解释 \\  
   & &  ★	能够完整感知文本的主要内容 \\    
\cline{2-3}

   &   \multirow{2}{*}{C}  & ★	不能对所提取的信息做出加工;或只能从一个文段中提取直接陈述的信息,利用直接提取的信息进行解释 \\  
   & &  ★	不能完整感知文本的主要内容 \\
   
\hline
   \multirow{9}{*}{习作}  & \multirow{3}{*}{A}  & ★	能够根据给定的情境叙述一件事情或介绍一种事物,想象有新意,内容丰满  \\  
   & &  ★	能够合理、巧妙安排故事内容 \\    
   & &  ★	语句通顺,描写生动,标点正确 \\    
\cline{2-3}      

   &   \multirow{3}{*}{B}  &  ★	能够根据给定的情境叙述一件事情或介绍一种事物,结构基本完整 \\  
   & & ★	能较清楚地表达自己的意思,语言较为通顺 \\  
   & &  ★	有个别错别字,标点符号使用基本正确 \\    
\cline{2-3}

   &   \multirow{3}{*}{C}  & ★	不能够根据给定的情境作文;或能够根据给定的情境作文,但结构不完整;或内容过于简单 \\  
   & &  ★	不能清楚地表达自己的意思 \\  
   & &  ★	错别字较多,标点符号使用不规范 \\ 
   
\hline 
\end{tabularx}
\end{table}
