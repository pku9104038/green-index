# 学业质量绿色指标测试的实施概况

根据本市中小学教育质量综合评价改革的要求,上海市教育委员会组织成立了上海市中小学学业质量绿色指标综合评价项目组(以下简称“项目组”),按照《上海市教育委员会关于做好2016年度上海市中小学学业质量绿色指标综合评价工作的通知》(沪教委基[2016]61号)的文件要求,组织了2016年度小学学业质量绿色指标测试,测试对象为在上海市普通中小学校就读的四年级学生,采用学科测试和问卷调查收集数据。项目组在评价队伍的建设、测试工具的研发、考务的组织与实施、评分标准的研制和阅卷、学业水平标准设定等方面开展了相关工作,以下是项目实施的概况。

## 测试概况

### 测试对象

2016年,全市16个区均抽样参加了“绿色指标”测试。普通民办学校自愿参加,所有公办小学全部参加本次调查。每一所参加调查的学校是根据其学校规模按照随机等距抽样的方式抽取一定数量的样本学生。

本次共有905所小学68046名四年级学生参加了学科测试和问卷调查,其中每名学生参加语文和数学两门学科的测试。6240位教师、1593位校长(含分管教学的副校长)、66347位学生家长分别参加了教师问卷、校长问卷及家长问卷的调查。

### 学科测试内容

`r tab_name <- "中小学生学业质量测试科目及测试框架"`
本次四年级测试涉及语文、数学两门学科。各学科测试的内容不仅包括学生在基础知识、基本技能方面所达到的水平,而且还包括时代发展所要求的小学生需具备的搜集处理信息、自主获取知识、分析与解决问题、交流与合作、创新精神与实践能力等核心素养。本次测试的学科测试框架见表\ref{tab: `r tab_name`}。

\begin{table}[H]
\centering
\caption{`r tab_name`} \label{tab: `r tab_name`} 
\itshape
\mytable
\begin{tabularx}{\textwidth}{|c|l|X|}
\hline

  & \makecell[c]{\mytablehead{内容}} & \makecell[c]{\mytablehead{能力}} \\
\hline

 \multirow{9}{3cm}{\makecell[cc]{语文}} & \multirow{3}{3cm}{积累} & 掌握已学过的常用汉字 \\
 & & 具有独立识字的能力 \\
 & & 积累古诗 \\
\cline{2-3}

 & \multirow{2}{3cm}{阅读} & 初步具有现代文阅读能力 \\
 & & 具备基本的语文积累和综合运用能力 \\
\cline{2-3}

 & \multirow{3}{3cm}{\makecell[cl]{习作}} & 能根据写作要求选材立意 \\
 & & 能根据表达的需要合理安排文章结构 \\
 & & 能文从字顺地表达自己的意思 \\
 & & 能正确、规范、端正地书写汉字 \\

\hline
\multirow{4}{3cm}{\makecell[cc]{数学}} & \multirow{4}{3cm}{\makecell[cl]{数与代数\\ 图形与几何\\ 统计与概率}} & 知识技能 \\
 & & 数学理解 \\
 & & 规则运用 \\
 & & 问题解决 \\
\hline

\end{tabularx}
\end{table}

### 问卷调查内容

本项目使用学生、教师、校长、家长问卷了解学生学习、教师专业发展、学校发展状况、家长对学校教育的感受状况及影响学生学业水平的相关因素。


学生问卷调查的内容包括学生基本情况、学校环境、学习压力、师生关系、学习动机、自信心和学习方式等。


教师问卷调查的内容包括教师基本情况如学历、任职经历、职称、工作感受情况、教师对学校教学管理的评价、教学观念和教师专业发展需求与困难等。


校长问卷调查的内容包括校长及学校基本情况、校长课程领导力、办学自主权、国家课程开设情况和对教师的专业支持等。


家长问卷调查的内容包括亲子关系、对学校教育的感受等。


## 工作流程

### 测试工具开发

以绿色指标测试的学科(包括背景问卷)为单位,组织了一支由市、区教研员、一线优秀教师组成的命题队伍。项目组负责把握命题的方向与总体框架,对命题过程进行管理与监督,保证试题的质量。
根据本次测试的目标和任务,确定了以下命题原则:


第一,命题基于课程标准,体现课程标准的基本要求和理念。试题注重考查学生对学科核心知识、技能的理解和掌握,尤其是学生综合运用所学知识解决实际问题的能力、收集与分析信息的能力以及对重要学科思想方法的理解与掌握。


第二,命题严格控制试卷的总体难度,组卷以体现课程标准对学生基本要求的题目为主,有难度的题目主要考查学生的高层次思维能力。


第三,试题形式为客观性试题和主观性试题相结合。试题多使用真实的情境和任务,注重通过客观性试题考察学生高层次思维能力。


`r tab_name <- "学科测试工具的研发过程" `
本项目采用规范、严谨的程序开展试题编制工作,以下是2016年度命题工作的过程及时间安排(见表\ref{tab: `r tab_name`})。

\begin{itshape}
\mytable
\begin{longtable}{m{13cm}r}
\caption{`r tab_name`} \label{tab: `r tab_name`} \\

\hline

\multicolumn{1}{c}{\multirow{2}{*}{\mytablehead{测试工具的研发}}} & \mytablehead{时间节点} \\
  & \mytablehead{(2016年)} \\
  
\hline

绿色指标项目组正式启动测试工具的研发工作 & 5月30日 \\
各学科命题组学习、研讨各学科测试框架,依据往年测试情况和2016年度工作需要开展命题和组卷工作   & 6月1-30日 \\
各学科命题组依据测试框架研制出四套试卷 & 6月30日前 \\
召开第一次专家审议会议,依据预测试数据分析结果及专家评审意见,对试卷进行审核和筛选,组成三套试卷 & 7月3-4日 \\
召开第二次专家审议会议,各学科命题组对试题试卷进行调整及修订 & 8月8-10日 \\
召开第三次专家审议会议,对各学科测试卷和背景问卷进行评审,各学科命题组根据评审意见对试题进行修订及优化 & 9月2-3日 \\
召开第四次专家审议会议,各学科命题组依据评审结果对试题进行优化及适应性调整,最终确定正式卷和备用卷,制作答题卡 & 10月7-10日 \\

\hline
\end{longtable}
\end{itshape}

### 施测

项目组编制了《2016年上海市中小学学业质量绿色指标综合评价实施手册》。上海市教委教研室、市教育督导部门等根据考务工作细则组织开展相关工作,包括编制培训手册,逐级对考务人员及管理人员进行培训,编制考生名册,制定详细的考务手册,组织具体的考务管理工作等。整个测试过程按照标准的考试进行管理,包括培训、考场设置、监考等环节。其中,所有参加测试学生名单在测试当天公布,参测学校负责人签订测试责任书。

### 评分标准研制

本次测试的非选择题均由人工批阅,因此评分标准的准确、全面和可操作性极其重要。在项目组的统一安排下,评分标准的研制由各学科命题组承担。在正式阅卷开始之前,各学科分别随机抽取适量份数,采取网上阅卷的方式,依据初步的评分标准对抽取到的学生作答进行评分。学科组经过反复讨论,修订、细化评分标准,最终形成评分标准。

### 阅卷

组建了由学科命题组成员、各区学科教研员及一线优秀教师组成的阅卷队伍。各学科命题组长担任阅卷组长,负责对阅卷人员进行培训和管理。整体阅卷工作原则上参照高考的相应要求组织实施,开放性试卷的评阅严格遵循国际上大型测试项目评卷标准进行,阅卷人员由具有较强责任心,丰富教学经验,相应职称水平,有大型测试阅卷工作经验的人员组成。


阅卷组长对阅卷人员进行的培训分为两部分,一是对新研制的评分标准进行培训,让阅卷员准确掌握自己需要评阅题目的评分要求;二是针对如何使用网络阅卷程序进行技术培训。


阅卷中采取的质量控制方式包括以下举措:


\mylist

- 在计算机阅卷系统中预设各题目的给分范围,由计算机自动判断评分是否超出范围,及时给予提醒,极大地降低了出错的可能性。
- 控制工作时间和工作进度,避免阅卷员由于阅卷速度过快或疲劳降低评分准确性。
- 阅卷组长对评分过程中出现的分歧组织讨论、统一认识和做法并对最终结果进行认定。
- 部分主观题采取多人共评,个体独立阅卷的方式,多人评分的平均值作为最终得分;当评分差值超过预先设定的阈值,由阅卷组长根据答卷情况对结果进行仲裁。
- 各学科阅卷组设质量监督员,全程关注由计算机生成的评分信息,包括评分一致性、阅卷工作进度、重评率等,发现异常现象及时通知相关阅卷人员。
发现异常现象及时通知相关阅卷人员。

\mytext

### 测试数据的录入、整理

学科测试卷采取填涂答题卡的方式,其录入由机器扫描完成。扫描答题卡时,计算机会根据预设条件对数据进行自动检测和判断,在这个过程中由专人进行全程监督,发现异常现象或错误数据立即进行纠正。

数据分析组根据一定的准则对数据进行清理,剔除空白卷、缺考卷、无考号卷的作答信息。若某份试卷的客观题作答信息全部为空或者主观题全部没有作答,并且经核查不是漏扫或漏评的视为无效作答。


本次背景问卷的信息反馈采用网络作答形式,学生、教师、校长及家长可用电脑或手机,在规定时间段内登录相应网址进行填写,提高了问卷调查的便捷性和数据采集的准确性。

### 学业水平标准设定

项目组邀请国际知名专家香港中文大学原副校长侯杰泰教授和美国ETS(美国教育考试服务中心)测量专家钱家和博士主持开展“学业水平标准设定”工作。各学科的专家评委由课程标准编制者、教材主编、区教研员、不同区域不同教龄的一线教师、学生家长等16人组成,采用Angoff和Bookmark的方法共同研讨和确定了上海地区相关学科的及格、良好和优秀的标准。

数据分析组对本次“绿色指标”测试结果进行了整理和分析,研讨了数据处理与等值方案,划定了2016年度上海市小学两门学科的学业水平。学业水平的划分结果采用A、B、C三个水平进行表述,其中A水平为最高等级,C水平为最低等级。

### 数据分析与报告撰写

本项目在测试数据分析的过程中,将经典测量理论与项目反应理论等先进测量技术相结合,对数据进行充分的分析,保证了测试结果分析的科学性。在此基础上,本次测试中所有的学科测试结果都采用量尺分数。量尺分数是根据学生的作答情况,采用项目反应理论模型将学生能力分数转换成的测验得分数,它具有不受测验题目差异和测验难度影响的特点,同时能与本项目前几年的测试结果进行比较。小学各学科整体上的平均分在500分左右,标准差在100左右,在各学科的各个维度上的平均分在300分左右,标准差在50左右。

经过绿色指标项目组的充分研讨,确定了反馈报告的整体撰写框架,各学科命题组长分别对各学科测试的结果进行了分析并对学生在典型试题上的表现进行了评价。问卷部分呈现了学生、教师、校长及家长的基本情况并分析了学生学业成绩与有关背景变量的关系。

### 系列报告构成

`r tab_name <- "测试报告的构成" `
本次系列报告主要由三种层次共计5种报告9个类型构成。报告充分考虑到不同使用者对于报告内容要求不同的问题,依此将报告划分综合报告和学科报告,具体见表\ref{tab: `r tab_name`}。

\begin{table}[H]
\centering

\caption{`r tab_name`} \label{tab: `r tab_name`} 
\itshape
\mytable

\begin{tabularx}{\textwidth}{|r|c|c|c|X|}

\hline
 \multicolumn{1}{|c|}{\multirow{2}{*}{\mytablehead{名称}}}  &  \multicolumn{3}{c|}{\mytablehead{报告层次}} &  \multicolumn{1}{c|}{\multirow{2}{*}{\mytablehead{对象}}}\\ 
\cline{2-4}
               &   \mytablehead{市级}  & \mytablehead{区级} & \mytablehead{学校} &  \\             
\hline 
绿色指标基础数据报告 & √  & √ & & \multirow{2}{*}{教育决策者、教科研人员等} \\
\cline{1-4}
绿色指标指数报告 & √  & √ & &  \\ 
\hline 
绿色指标学校报告 &   &  & √ & 校长、教师等 \\ 
\hline 
小学语文学科报告 & √  & √ & & \multirow{2}{*}{学科教研员、教师等} \\
\cline{1-4}
小学数学学科报告 & √  & √ & &  \\
\hline 
\end{tabularx}
\end{table}
