
\usepackage{xeCJK} 
\setCJKmainfont[BoldFont={Heiti SC},ItalicFont={Kaiti SC},BoldItalicFont={Songti SC}]{STFangsong}
\setCJKsansfont{Songti SC}
\setCJKmonofont{Yuanti SC}
\setmainfont{Times New Roman}

\usepackage{fancyhdr}
\usepackage{ragged2e}
\usepackage{indentfirst}
\usepackage{float}

\geometry{a4paper}



\usepackage{tocloft}
\renewcommand\cftchapfont{\large\bfseries}
\renewcommand\cftchappagefont{\large\bfseries}



%\usepackage{mytitlesec}
%\titleformat{\chapter}{\center\Huge\bfseries}{~\,\thechapter\,~.}{1em}{}

\usepackage[fntef]{ctexcap}
\CTEXsetup[name={,},number={\chinese{chapter}},format+={\flushleft}]{chapter}
\CTEXsetup[beforeskip={0pt},afterskip={20pt}]{chapter}
%\CTEXsetup[name={第,章},number={\chinese{chapter}}]{chapter}
%\CTEXsetup[name={第,节},number={\chinese{section}}]{section}
%\CTEXsetup[number={\chinese{subsection}}]{subsection}
%\CTEXsetup[name={(,)},number={\chinese{subsubsection}}]{subsubsection}

\CTEXsetup[number={\chinese{chapter}}]{chapter}
\CTEXsetup[number={\chinese{section}}]{section}


%\titlespacing{\chapter}{0pt}{0pt}{0pt}

\makeatletter

\renewcommand\section{\@startsection {section}{2}{\z@}{-.5ex \@plus -.2ex \@minus -.2ex}{.5ex \@plus.2ex \@minus .2ex}{\normalfont\LARGE\bfseries}}

\renewcommand\subsection{\@startsection {subsection}{3}{\z@}{-.4ex \@plus -.1ex \@minus -.1ex}{.4ex \@plus.1ex \@minus .1ex}{\normalfont\large}}

\renewcommand\subsubsection{\@startsection {subsubsection}{4}{\z@}{-.3ex \@plus -.1ex \@minus -.1ex}{.3ex \@plus.1ex \@minus .1ex}{\normalfont\large}}

\renewcommand\paragraph{\@startsection {paragraph}{5}{\z@}{-.2ex \@plus -.1ex \@minus -.1ex}{.2ex \@plus.1ex \@minus .1ex}{\normalfont\large}}

\renewcommand\subparagraph{\@startsection {subparagraph}{6}{\z@}{-.1ex \@plus -1ex \@minus -.1ex}{.1ex \@plus.1ex \@minus .1ex}{\normalfont\large}}


%\newcommand\prefix@chapter{第 \thechapter 章 }

\makeatother

\renewcommand*\contentsname{目~录}

\usepackage{makecell,multirow,diagbox}

\usepackage{longtable}
\usepackage{lipsum} % just for dummy text- not needed for a longtable
%\usepackage{booktabs}
\renewcommand{\arraystretch}{1.5}
\renewcommand{\tablename}{表}


\renewcommand{\figurename}{图}


\usepackage{pdflscape}
\makeatletter
\global\let\orig@begin@landscape=\landscape%
\global\let\orig@end@landscape=\endlandscape%
\gdef\@true{1}
\gdef\@false{0}
\gdef\landscape{%
    \global\let\within@landscape=\@true%
    \orig@begin@landscape%
}%
\gdef\endlandscape{%
    \orig@end@landscape%
    \global\let\within@landscape=\@false%
}%
\@ifpackageloaded{pdflscape}{%
    \gdef\pdf@landscape@rotate{\PLS@Rotate}%
}{
    \gdef\pdf@landscape@rotate#1{}%
}
\let\latex@outputpage\@outputpage
\def\@outputpage{
    \ifx\within@landscape\@true%
        \if@twoside%
            \ifodd\c@page%
                \gdef\LS@rot{\setbox\@outputbox\vbox{%
                    \pdf@landscape@rotate{+90}%
                    \hbox{\rotatebox{90}{\hbox{\rotatebox{0}{\box\@outputbox}}}}}%
                }%
            \else%
                \gdef\LS@rot{\setbox\@outputbox\vbox{%
                    \pdf@landscape@rotate{+90}%
                    \hbox{\rotatebox{90}{\hbox{\rotatebox{0}{\box\@outputbox}}}}}%
                }%
            \fi%
        \else%
            \gdef\LS@rot{\setbox\@outputbox\vbox{%
                \pdf@landscape@rotate{+90}%
                \hbox{\rotatebox{90}{\hbox{\rotatebox{0}{\box\@outputbox}}}}}%
            }%
        \fi%
    \fi%
    \latex@outputpage%
}
\makeatother



\numberwithin{figure}{chapter}
\renewcommand{\thefigure}{\arabic{chapter}-\arabic{figure}}

\numberwithin{table}{chapter}
\renewcommand{\thetable}{\arabic{chapter}-\arabic{table}}


\newcommand{\mytitle}[1]{
    \begin{center}\fontsize{26pt}{39pt}\selectfont#1\end{center} 
}

\newcommand{\mysubtitle}[1]{
    \begin{center}\fontsize{18pt}{27pt}\selectfont\textbf{\textit{#1}}\end{center} 
}

\newcommand{\myproducer}[1]{
    \begin{center}\fontsize{14pt}{21pt}\selectfont\textbf{#1}\end{center} 
}

\newcommand{\mycopyright}[1]{
    \begin{center}\begin{bf}\fontsize{10.5pt}{18pt}\selectfont#1\end{bf}\end{center} 
}

\newcommand{\myprefacetitle}[1]{
    \begin{center}\begin{bf}\fontsize{24pt}{36pt}\selectfont#1\end{bf}\end{center} 
}

\newcommand{\myprefacetext}{
    \raggedright\fontsize{13.75pt}{20pt}\selectfont\setlength{\parindent}{2em}
    \setlength{\parskip}{1ex} 
}

\newcommand{\mytext}{
    \raggedright\small\setlength{\parindent}{2em}
    \setlength{\parskip}{1ex} 
}

\setlength{\abovecaptionskip}{0pt}
\setlength{\belowcaptionskip}{0pt}

%\usepackage{caption2} 
%\renewcommand{\captionfont}{\textit \large } 
%\renewcommand{\captionlabelfont}{\emph \large \centering} 

\usepackage[font=large]{caption}
\captionsetup[figure]{labelfont=bf,textfont=it,labelsep=space}
\captionsetup[table]{labelfont=bf,textfont=bf,singlelinecheck=off,justification=raggedright,labelsep=space}

\usepackage{tabularx}

\usepackage{ltxtable}

\usepackage[normalem]{ulem}

\usepackage{graphicx}
\usepackage{subcaption}
\usepackage{wrapfig}

\usepackage[bottom]{footmisc}

\raggedbottom
