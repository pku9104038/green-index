
\usepackage{xeCJK} 
\usepackage{fontspec}
%\setCJKmainfont[AutoFakeBold = false, ItalicFont={Kaiti SC},BoldItalicFont={Songti SC}]{STFangsong}
\setCJKmainfont[BoldFont={Songti SC},ItalicFont={STKaiti},BoldItalicFont={Songti SC}]{STFangsong}
\setCJKsansfont{Kaiti SC}
\setCJKmonofont{Yuanti SC}
\setmainfont{Yuanti SC}
%\setmainfont{Times New Roman}

\usepackage{fancyhdr}
\usepackage{ragged2e}
\usepackage{indentfirst}
\usepackage{float}
\geometry{a4paper}

\usepackage{tocloft}
\renewcommand\cftchapfont{\large\bfseries}
\renewcommand\cftchappagefont{\large\bfseries}


%\CTEXsetup[beforeskip={0pt},afterskip={20pt}]{chapter}

\usepackage{mytitlesec}
\titleformat{\chapter}{\centering\Huge\bfseries}{~\,\thechapter\,~.}{1em}{} 


\usepackage[fntef]{ctexcap}
\CTEXsetup[number={\chinese{chapter}}]{chapter}
\CTEXsetup[number={\chinese{section}}]{section}
%\CTEXsetup[number={\chinese{subsection}}]{subsection}
%\CTEXsetup[number={\chinese{subsubsection}}]{subsubsection}
%\CTEXsetup[number={\chinese{paragraph}}]{paragraph}
%\CTEXsetup[number={\chinese{subparagraph}}]{subparagraph}

\makeatletter

\renewcommand\section{\@startsection {section}{2}{\z@}{-.5ex \@plus -.1ex \@minus -.1ex}{.5ex \@plus.1ex \@minus .1ex}{\normalfont\fontsize{20pt}{36pt}\selectfont\bfseries}}

\renewcommand\subsection{\@startsection {subsection}{3}{\z@}{-.3ex \@plus -.1ex \@minus -.1ex}{.3ex \@plus.1ex \@minus .1ex}{\normalfont\fontsize{14pt}{24pt}\selectfont}}

\renewcommand\subsubsection{\@startsection {subsubsection}{4}{\z@}{-.1ex \@plus -.1ex \@minus -.1ex}{.1ex \@plus.1ex \@minus .1ex}{\normalfont\fontsize{12pt}{20pt}\selectfont}}

\renewcommand\paragraph{\@startsection {paragraph}{5}{\z@}{-.1ex \@plus -.1ex \@minus -.1ex}{.1ex \@plus.1ex \@minus .1ex}{\normalfont\fontsize{12pt}{18pt}\selectfont}}

\renewcommand\subparagraph{\@startsection {subparagraph}{6}{\z@}{-.1ex \@plus -1ex \@minus -.1ex}{.1ex \@plus.1ex \@minus .1ex}{\normalfont\fontsize{12pt}{18pt}\selectfont}}

%\newcommand\prefix@chapter{第 \thechapter 章 }

\makeatother

\renewcommand*\contentsname{目~~~~~录}

\usepackage{makecell,multirow,diagbox}

\usepackage{longtable}
\usepackage{lipsum} % just for dummy text- not needed for a longtable
%\usepackage{booktabs}
\renewcommand{\arraystretch}{1.5}
\renewcommand{\tablename}{表}


\renewcommand{\figurename}{图}


\usepackage{pdflscape}
\makeatletter
\global\let\orig@begin@landscape=\landscape%
\global\let\orig@end@landscape=\endlandscape%
\gdef\@true{1}
\gdef\@false{0}
\gdef\landscape{%
    \global\let\within@landscape=\@true%
    \orig@begin@landscape%
}%
\gdef\endlandscape{%
    \orig@end@landscape%
    \global\let\within@landscape=\@false%
}%
\@ifpackageloaded{pdflscape}{%
    \gdef\pdf@landscape@rotate{\PLS@Rotate}%
}{
    \gdef\pdf@landscape@rotate#1{}%
}
\let\latex@outputpage\@outputpage
\def\@outputpage{
    \ifx\within@landscape\@true%
        \if@twoside%
            \ifodd\c@page%
                \gdef\LS@rot{\setbox\@outputbox\vbox{%
                    \pdf@landscape@rotate{+90}%
                    \hbox{\rotatebox{90}{\hbox{\rotatebox{0}{\box\@outputbox}}}}}%               
                    %\pdf@landscape@rotate{-90}%
                    %\hbox{\rotatebox{90}{\hbox{\rotatebox{180}{\box\@outputbox}}}}}%
                }%
            \else%
                \gdef\LS@rot{\setbox\@outputbox\vbox{%
                    \pdf@landscape@rotate{+90}%
                    \hbox{\rotatebox{90}{\hbox{\rotatebox{0}{\box\@outputbox}}}}}%

%                    \pdf@landscape@rotate{+90}%
%                    \hbox{\rotatebox{90}{\hbox{\rotatebox{0}{\box\@outputbox}}}}}%
                }%
            \fi%
        \else%
            \gdef\LS@rot{\setbox\@outputbox\vbox{%
                \pdf@landscape@rotate{+90}%
                \hbox{\rotatebox{90}{\hbox{\rotatebox{0}{\box\@outputbox}}}}}%
            }%
        \fi%
    \fi%
    \latex@outputpage%
}
\makeatother



\numberwithin{figure}{chapter}
\renewcommand{\thefigure}{\arabic{chapter}-\arabic{figure}}

\numberwithin{table}{chapter}
\renewcommand{\thetable}{\arabic{chapter}-\arabic{table}}


\newcommand{\mytitle}[1]{
    \begin{center}\fontsize{27.5pt}{41.25pt}\selectfont#1\end{center} 
}

\newcommand{\mysubtitle}[1]{
    \begin{center}\fontsize{24pt}{48pt}\selectfont\textbf{\textit{#1}}\end{center} 
}

\newcommand{\myproducer}[1]{
    \begin{center}\fontsize{14pt}{21pt}\selectfont\textbf{#1}\end{center} 
}

\newcommand{\mycopyright}[1]{
    \begin{center}\begin{bf}\fontsize{10.5pt}{15.75pt}\selectfont#1\end{bf}\end{center} 
}

\newcommand{\myprefacetitle}[1]{
    \begin{center}\begin{bf}\fontsize{24pt}{72pt}\selectfont#1\end{bf}\end{center} 
}

\newcommand{\myprefacetext}{
    \raggedright\fontsize{13.75pt}{20.625pt}\selectfont\setlength{\parindent}{2em}
    \setlength{\parskip}{0.75ex} 
}

\newcommand{\mytext}{
    \raggedright\fontsize{12pt}{18pt}\selectfont\setlength{\parindent}{2em}
    \setlength{\parskip}{0.5ex} 
}

\newcommand{\questiontitle}[1]{
    \begin{bf}\raggedright\fontsize{14pt}{18pt}\selectfont#1\end{bf}
}

\newcommand{\myquot}{
    \raggedright\fontsize{12pt}{18pt}\selectfont\setlength{\parindent}{2em}
}

\newcommand{\mytablehead}[1]{
    \fontsize{12pt}{12pt}\selectfont#1
}
\newcommand{\mytableheadfont}{
    \fontsize{12pt}{12pt}\selectfont
}

\newcommand{\mytable}{
    \fontsize{10.5pt}{10.5pt}\selectfont
}

\newcommand{\mylist}{
    \fontsize{10.5pt}{16pt}\selectfont
}

\setlength{\abovecaptionskip}{0pt}
\setlength{\belowcaptionskip}{0pt}

%\usepackage{caption2} 
%\renewcommand{\captionfont}{\textit \large } 
%\renewcommand{\captionlabelfont}{\emph \large \centering} 

\usepackage[font=normalsize]{caption}
\captionsetup[figure]{labelfont=it,textfont=it,labelsep=space}
\captionsetup[table]{labelfont=bf,textfont=bf,singlelinecheck=off,justification=raggedright,labelsep=space}

\usepackage{tabularx}
\usepackage{ltxtable}

\usepackage[normalem]{ulem}

\usepackage{graphicx}
\usepackage{subcaption}
\usepackage{wrapfig}

\usepackage[bottom]{footmisc}
\raggedbottom
